\FILE{rest.tex}

\section{REST}\label{rest}

Cloudmesh cc can be interfaced with Representational State Transfer
(REST) by using Curl commands. REST is advantageous as it can be used
with programming languages other than Python.

To use this feature, ensure that the FastAPI server is online with the
following commands:

\begin{minted}[breaklines]{bash}
$ cd ~/cm/cloudmesh-cc
$ cms cc start
\end{minted}

\subsection{List Workflows Available on Local
Computer}\label{list-workflows-available-on-local-computer}

The following command returns a dict that lists each available workflow.

\begin{minted}[breaklines]{bash}
curl -X 'GET' 'http://127.0.0.1:8000/workflows' -H 'accept: application/json'
\end{minted}

\subsection{Upload a Workflow}\label{upload-a-workflow}

There are different ways to upload a new workflow to cloudmesh-cc, by
specifying one of the following:

\smallskip
\begin{itemize}
\item
  a directory that contains the scripts and YAML
\end{itemize}

\begin{minted}[breaklines]{bash}
curl -X 'POST' 'http://127.0.0.1:8000/workflow?\
directory=~/cm/cloudmesh-cc/tests/workflow-example' -H 'accept: application/json' -d ''
\end{minted}
\smallskip

\smallskip
\begin{itemize}
\item
  an archive file in tgz, xz, tar.gz, or tar format that contains the
  scripts and yaml
\end{itemize}

\begin{minted}[breaklines]{bash}
curl -X 'POST' 'http://127.0.0.1:8000/workflow?\ 
archive=ThePathToYourArchiveFile' -H 'accept: application/json' -d ''
\end{minted}
\smallskip

\smallskip
\begin{itemize}
\item
  or a yaml file that specifies what jobs to run.
\end{itemize}

\begin{minted}[breaklines]{bash}
curl -X 'POST' 'http://127.0.0.1:8000/workflow?\
yaml=~/cm/cloudmesh-cc/tests/workflow-example\
/workflow-example.yaml' -H 'accept: application/json' -d ''
\end{minted}
\smallskip

\subsection{Delete a Workflow}\label{delete-a-workflow}

The following command deletes a preexisting workflow from the local
computer.

To use it, replace \texttt{workflow-example} with the workflow you wish
to delete.

\begin{minted}[breaklines]{bash}
curl -X 'DELETE' 'http://127.0.0.1:8000\
/workflow/workflow-example' -H 'accept: application/json'
\end{minted}

\subsection{Retrieve a Workflow}\label{retrieve-a-workflow}

The following command retrieves a workflow and its configuration, such
as the directory where it resides, the jobs that it contains, the colors
of the statuses, and more.

To use it, replace \texttt{workflow-example} with the workflow you wish
to retrieve.

\begin{minted}[breaklines]{bash}
curl -X 'GET' 'http://127.0.0.1:8000/workflow\
/workflow-example' -H 'accept: application/json'
\end{minted}

\subsection{Run a Workflow}\label{run-a-workflow}

The following command runs a workflow.

To use it, replace \texttt{workflow-example} with the workflow you wish
to run. Additionally, to hide the graph of the workflow as it is run,
replace \texttt{show=True} with \texttt{show=False}.

\begin{minted}[breaklines]{bash}
curl -X 'GET' 'http://127.0.0.1:8000/workflow\
/run/workflow-example?show=True' -H 'accept: application/json'
\end{minted}

\subsection{Add a Job to a Workflow}\label{add-a-job-to-a-workflow}

The following command adds a new job to a preexisting workflow.

The only required parameter is the job parameter, which specifies the
name of the job. The other optional parameters include the user, host,
label, kind, status, progress, and script. To use the command, replace
\texttt{workflow-example} with the workflow you wish to change.
Additionally, change the other parameters, as desired.

\begin{minted}[breaklines]{bash}
curl -X 'POST' 'http://127.0.0.1:8000/workflow\
/job/workflow-example?job=myJob&user=aPerson\
&host=local&kind=local&status=ready&\
script=aJob.sh&progress=0&label=aLabel' -H 'accept: application/json'
\end{minted}
