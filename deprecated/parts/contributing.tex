\FILE{contributing.tex}

\subsection{Contributing}\label{contributing}

All contributions are done under the Apache License.  The code is
maintained as an Open Source project on GitHub while using the typical
GitHub code management tools such as

\begin{itemize}
\item
  \href{https://github.com/cloudmesh/cloudmesh-cc}{Code Repository}
\item
  \href{https://github.com/cloudmesh/cloudmesh-cc/issues}{Issue Management}
\item
  \href{https://github.com/cloudmesh/cloudmesh-cc/pulls}{Pull Request Management}
\item
  \href{https://github.com/cloudmesh/cloudmesh-cc/actions}{Automatic verification with GitHub Actions}
\end{itemize}

The main branch is the release branch and is supposed to be functional
at all times. Hence, contributions are first done in other branches,
and once agreeing that they need to be integrated into the code, they
are merged into main. All new code must be documented and have
sufficient automated tests.  Before creating a pull request it is
important that the tests within the test directory are passing.  The
repository already contains several pytests that can be leveraged to
conduct routine testing of the code, including its SSH remote
functionality, REST capability, and Python interface, among others.
