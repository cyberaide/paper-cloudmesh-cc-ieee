\appendix

\section{Appendix}

\subsection{Installation}\label{installation}

To leverage cloudmesh-cc, use the cloudmesh-installer to install the
Cloudmesh suite of repositories. Optionally, to utilize the graph
visualization you must install also {\em graphviz} must be
installed. On Windows, Git Bash is required, in addition. The overall
installation is very simple and is supported on a variety of operating
systems. We leverage the cloudmesh-installer to locally install the
cloudmesh suite of repositories by executing the following commands:

{\scriptsize\begin{verbatim}
  $ mkdir ~/cm
  $ cd ~/cm
  $ pip install cloudmesh-installer -U
  $ cloudmesh-installer get cc
\end{verbatim}}

To install graphviz you can use the following commands on the
appropriate operating system:

\begin{description}

\item[Windows.]  Git Bash and Graphviz must be installed. The user
can use an instance of Chocolatey that is run as an administrator for
convenience:

{\scriptsize\begin{verbatim}
  $ choco install git.install
          --params "/GitAndUnixToolsOnPath
                    /Editor:Nano
                    /PseudoConsoleSupport
                    /NoAutoCrlf" -y
  $ choco install graphviz -y
\end{verbatim}}

\item[macOS.] Graphviz must be installed. The user can use Homebrew
for convenience:

{\scriptsize\begin{verbatim}
  $ brew install graphviz
\end{verbatim}}

\item[Linux.] Graphviz must be installed. The user can use apt for
convenience:

{\scriptsize\begin{verbatim}
  $ sudo apt install graphviz -y
\end{verbatim}}

\end{description}

To test the workflow program, prepare a {\scriptsize \verb|cm|}
directory in your home directory by executing the following commands:

{\scriptsize\begin{verbatim}
  $ cd ~/cm/cloudmesh-cc
  $ pytest -v -x --capture=no tests
\end{verbatim}}

A variety of separate tests are available that test individual
capabilities.


% \FILE{cloudmask-appendix.tex}

\subsection{Running Cloudmask Workflow}\label{sec:running-cloudmask}

To run the Cloudmask workflow, run the following commands:

\begin{minted}[breaklines]{bash}
$ cd ~/cm
$ git clone https://github.com/laszewsk/mlcommons.git
$ cd mlcommons
$ pytest -v -x --capture=no benchmarks/cloudmask/target/rivanna/run_cloudmask_workflow.py
\end{minted}


\FILE{cloudmask-appendix.tex}

\subsection{Running Cloudmask Workflow}\label{sec:running-cloudmask}

To run the Cloudmask workflow, run the following commands:

{\scriptsize\begin{verbatim}
  $ cd ~/cm
  $ git clone https://github.com/laszewsk/mlcommons.git
  $ cd mlcommons
  $ pytest -v -x --capture=no
           benchmarks/cloudmask/target/rivanna/run_cloudmask_workflow.py
\end{verbatim}}


\FILE{contributing.tex}

\subsection{Contributing}\label{contributing}

All contributions are done under the Apache License. The code is
maintained as an open-source project on GitHub while using the typical
GitHub code management tools such as:

\begin{itemize}
\item
  \href{https://github.com/cloudmesh/cloudmesh-cc}{Code Repository}
\item
  \href{https://github.com/cloudmesh/cloudmesh-cc/issues}{Issue Management}
\item
  \href{https://github.com/cloudmesh/cloudmesh-cc/pulls}{Pull Request Management}
\item
  \href{https://github.com/cloudmesh/cloudmesh-cc/actions}{Automatic verification with GitHub Actions}
\end{itemize}

The main branch is the release branch and is supposed to be functional
at all times. Hence, contributions are first done in other branches,
and once agreeing that they need to be integrated into the code, they
are merged into main. All new code must be documented and have
sufficient automated tests. Before creating a pull request, it is
important that the tests within the test directory are passing. The
repository already contains several pytests that can be leveraged to
conduct routine testing of the code, including its SSH remote
functionality, REST capability, and Python interface, among others.

